\chapter{Planungsphase}

\section{Prozessablauf}

Die folgende \literef{Ablaufdiagramm} stellt den gesamten Prozess als Ablaufdiagramm dar.
\bild[0.65]{Ablaufdiagramm.pdf}{Ablauf des Benachrichtigungsprozesses}{Ablaufdiagramm}


\section{Server}
Damit gewährleistet werden kann, dass der Service die ganze Zeit zur Verfügung steht, wird ein Server benötigt.
Auf diesem Server werden alle benötigten Software sowie Mittelwertapplikationen installiert und eingerichtet.
Für das folgende Projekt wird ein Linux-Server bei dem deutschen Serverhosting-Unternehmen \textit{Netcup} angemietet\vglink{netcup.de}{https://www.netcup.de/}.

\subsection{Betriebssystem}
Als Betriebssystem wird eine \textit{Rocky Linux} Distribution verwendet.
\textit{Rocky Linux} ist eine Linux-Distribution und der inoffiziellen Nachfolger der \textit{CentOS 8} Distribution.
Das Betriebssystem \textit{Rocky Linux} basiert auf der Linux-Distribution \textit{\ac{RHEL8}}.
Die Gründe für die Entscheidung der Verwendung des Betriebssystems \textit{Rocky Linux} sind vielfältig.
Wie bereits erwähnt ist das Betriebssystem \textit{Rocky Linux} der inoffizielle Nachfolger des Betriebssystems \textit{CentOS 8} und damit kompatibel mit dem Betriebssystem \textit{\ac{RHEL8}}.
Das bedeutet, dass alle Funktionen und Einstellungen, welche bei dem Betriebssystem \textit{\ac{RHEL8}} funktionieren, direkt auf dem Betriebssystem \textit{Rocky Linux} übernommen werden können.
Die große Community, welche hinter dem Betriebssystem \textit{Rocky Linux} steht, sorgt dabei für eine enorme Ansammlung von zur Verfügung stehenden Wissen.
Das von der Community bereitgestellte Wissen, ist enorm hilfreich, um bei etwaigen Problemen Lösung zu finden.
Ein weiterer Vorteil, den das Betriebssystem \textit{Rocky Linux} bietet, ist der spezielle Focus auf Stabilität.
Anderes als andere Linux-Distribution ist das Betriebssystem \textit{Rocky Linux} ebenso wie die Ursprungs-Distribution \textit{\ac{RHEL8}} nicht auf Innovation, sondern auf Stabilität ausgelegt.
Dazu gehört auch die Langzeitpflege der \ac{LTS} Version.\vgonlinezitat{RHEL8}

\section{Webseite}
Damit die Studenten der BA-Glauchau den Service zur automatischen E-Mail Benachrichtigung neuer Campus-Dual Noten nutzen können, wird eine Webseite benötigt, auf welcher sich die Studenten registrieren können.
Dafür müssen sich ein Student zunächst auf der Webseite registrieren und anschließend auf der Webseite anmelden.
Der Student hat nun die Möglichkeit den Benachrichtigungsservice zu aktivieren, zu deaktivieren oder auch seine gesamten hinterlegten Daten zu löschen.
Werden alle Daten unwiderruflich gelöscht, ist die Registrierung hinfällig und der Student hat keinen Zugriff mehr auf den Service.
Für das Projekt wird daher eine Webseite für die Registrierung, eine Webseite für den Login und eine Webseite für die Service-Bedienung benötigt.

\section{Software}

\subsection{MariaDB}
Für die Speicherung der Daten der einzelnen Studenten wird auf eine \textit{Mysql8-Datenbank} zurückgegriffen.
Die \textit{Mysql8-Datenbank} wird direkt auf dem \textit{Rocky Linux Server} installiert und eingerichtet.
Ein Datensatz enthält dabei alle Informationen eines einzelnen Studenten.
Jedem Student werden die folgenden Attribute zugeordnet.

\textbf{ID:}\\
Die ID wird zur Identifizierung eines einzelnen Datensatzes benötigt.
Dieser erhält die zusätzliche Eigenschaft eines Primärschlüssels und darf nur einmal auftreten.

\textbf{Name:}\\
Der Name dient zur Identifizierung des einzelnen Studenten.
Dieser wird benötigt, um den Studenten mit dem entsprechenden Namen in der versendeten E-Mail anzusprechen.
Für den eigentlichen Prozess ist dieser nicht von Bedeutung und kann von daher frei gewählt werden.

\textbf{Email:}\\
Die E-Mail wird für die Registrierung sowie für den Login benötigt.
Es muss sich um eine gültige E-Mail handeln, da die Benachrichtigung einer neuen Note im Verwaltungstool \textit{Campus Dual} über an diese E-Mail versendet wird.

\textbf{Matrikelnummer:}\\
Die Matrikelnummer ist die eindeutige Identifikationsnummer des einzelnen Studenten und muss korrekt angegeben werden.
Diese wird später benötigt, um an die von \textit{Campus Dual} bereitgestellte \ac{JSON} Datei heranzukommen, in welcher sich die Informationen der neuen Noten befinden.

\textbf{Passwort:}\\
Das Passwort ist das Kennwort, mit welchem sich der Student auf der Webseite anmelden kann, um den Service zu aktivieren, zu deaktivieren bzw. seinen gesamten Datensatz zu löschen.
Aus Sicherheitsgründen wird das Passwort nicht im Klartext gespeichert, sondern wird mit einer Hashfunktion verschleiert.

\textbf{Hashwert:}\\
Der Hashwert ist individuell für jeden einzelnen Studenten und muss korrekt angegeben werden.
Eine Anleitung, wie der einzelne Hashwert herausgefunden werden kann, wird in einem späteren Kapitel näher erläutert.
Der Hashwert wird benötigt, um an die von \textit{Campus Dual} bereitgestellte \ac{JSON} Datei heranzukommen, in welcher sich die Informationen der neuen Noten befinden.

\textbf{Benachrichtigung:}\\
Die Benachrichtigung ist ein boolescher Wert und gibt, an, ob der Student eine E-Mail Benachrichtigung erhalten möchte, falls eine neue Note in \textit{Campus Dual} verfügbar ist.

\textbf{Letzter Notenstand:}\\
Damit die spätere Software den aktuellen und neuen Notenstand vergleichen kann, wird ein Attribut benötigt, welches den stets aktuellen Notenstand des Studenten beinhaltet.

\subsection{Apache}
Für die Bereitstellung eines Webservices wird auf einen Apache-Server zurückgegriffen.
Der Apache-Server wird direkt auf dem \textit{Rocky Linux Server} installiert.
Eine detaillierte Anleitung der Installation und Konfiguration des Apache-Servers wird in einem späteren Kapitel näher erläutert.
Um die Sicherheit der übertragenen Informationen vom Client zum Server zu erhöhen, wird der Apache-Server mit einem zusätzlich mit Zertifikat ausgestattet.

\subsection{phpmysql}
Das Verwaltungstool \textit{phpmyadmin} dient zur erleichterten Bedienung und Verwaltung der \textit{Mysql8-Datenbank}.
Dabei können Datenbanken, Tabellen und Werte übersichtlich und sicher gewartet werden.
Alternativ besteht die Möglichkeit, die \textit{Mysql8-Datenbank} auch direkt über den \textit{Rocky Linux Server} anzusprechen und einzurichten.
Eine detaillierte Einleitung zur Installation sowie Konfiguration des Verwaltungstools \textit{phpmyadmin} wird in den späteren Kapiteln näher erläutert.
Um die Sicherheit zu erhöhen, wird der Zugriff auf die Datenbank über das Verwaltungstool \textit{phpmyadmin} nur über spezielle IP-Adressen erlaubt.

%\subsection{Software zum Auswerten neuer Notesn}

