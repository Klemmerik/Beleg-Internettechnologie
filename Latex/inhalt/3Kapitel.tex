\chapter{Installation und Konfiguration}
\section{Rocky-Linux Distribution}
Um den Server wie gewünscht nutzen zu können, sind einige Vorbereitungsschritte für die Grundkonfiguration der \textit{Rocky Linux Distribution} auszuführen.
Dazu gehört die eigentliche Installation der Distribution sowie weitere Server interne Konfigurationen.
Das Betriebssystem wird als Minimalkonfiguration installiert\fn{Eine Server-Minimalinstallation dient lediglich der Installation des Betriebssystems ohne weitere bzw. ungewünschter Softwareapplikationen.}.
Nach der erfolgreichen Installation des Betriebssystems ist der Server einsatzbereit.
Es werden die folgenden Serverkonfigurationen durchgeführt:

\textbf{SSH-Zugänge}

Das Arbeiten auf dem Server ist essenziell.
Die von \textit{NetCup} zur Verfügung gestellte Kommandozeilenkonsole ist jedoch nicht für effizientes Arbeiten ausgelegt.
Diese dient ausschließlich für den grundlegenden Zugriff des neu aufgesetzten Servers und bietet keine weiteren Funktionen wie beispielsweise \textit{Copy-and-paste} oder das Importieren von Dateien.
Um über Dritt-Anbieter, wie beispielsweise \textit{MobaXterm} auf den Server zuzugreifen, wird daher für jeden Entwickler ein entsprechender Nutzer mit \ac{SSH}-Zugriffsrechten benötigt.
Für das folgende Projekt werden die Nutzer \textit{til} und \textit{erik} angelegt.
Die Nutzer werden mit den folgenden Kommandozeilenbefehlen aus \literef{Linux-Benutzer anlegen} erstellt und anschließend mit Root-Rechten versehen.\vgonlinezitat{user}

\begin{code}
    \begin{minted}{bash}
        # neuen Benutzer til erstellen
        adduser til
        # neuen Benutzer erik erstellen
        adduser erik

        # neues Passwort für Benutzer til festlegen
        passwd til
        # neues Passwort für Benutzer erik festlegen
        passwd erik

        # Benutzer til in die wheel Gruppe (root) hinzufügen
        usermod -aG wheel til
        # Benutzer erik in die wheel Gruppe (root) hinzufügen
        usermod -aG wheel erik
    \end{minted}
    \caption[Linux-Benutzer anlegen]{Linux-Benutzer anlegen}
    \label{Linux-Benutzer anlegen}
\end{code}

\newpage

Nachdem die Benutzer angelegt und mit den entsprechenden Rechten versehen wurden, werden die \ac{SSH}-Schlüssel eingerichtet.
Für die Generierung neuer \ac{SSH}-Schlüssel steht das Linux-Tool \textit{ssh-keygen} zur Verfügung, welches standardmäßig in der Minimalinstallation enthalten ist.
Das Tool \textit{ssh-keygen} erzeugt dabei einen privaten sowie öffentlichen \ac{SSH}-Schlüssel.
Die \ac{SSH}-Schlüssel werden mit dem folgenden Kommandozeilenbefehlen generiert und erzeugen die folgende Ausgabe - \fullref{Generierung der SSH-Schlüssel}.\vgonlinezitat{ssh}
\begin{code}
    \begin{minted}{bash}
        # mit den Benutzer til
        ssh-keygen
        # Ausgabe auf der Konsole
        Generating public/private rsa key pair.Enter file in which to save the key (/home/til/.ssh/id_rsa):

        # mit den Benutzer erik
        ssh-keygen
        # Ausgabe auf der Konsole
        Generating public/private rsa key pair.Enter file in which to save the key (/home/erik/.ssh/id_rsa): 
    \end{minted}
    \caption[Generierung der SSH-Schlüssel]{Generierung der SSH-Schlüssel}
    \label{Generierung der SSH-Schlüssel}
\end{code}

Mit den hinterlegten \ac{SSH}-Schlüssel der Benutzer ist es nun möglich, eine sichere SSH-Verbindung zu dem Server aufzubauen.

\textbf{Server-Zugang beschränken}

Aus Sicherheitsgründen wird der Zugang auf den Server beschränkt.
Befugte Nutzer soll es nur möglich sein, mit einer sicheren SSH-Verbindung auf den Server zuzugreifen.
Eine Verbindung mit dem Server mittels Passwort wird verboten.
Zudem ist eine remote Verbindung mit dem Root-Benutzer nicht möglich.
Dafür wird die folgende Konfigurationdatei wie angepasst:

\mintinline{bash}{/etc/ssh/sshd_config}

Die Konfigurationsdatei enthält Einstellungen, welche die SSH-Zugriffsrechte des Servers abbilden.
In der oben aufgezeigten Konfigurationdatei werden die folgen zwei Zeilen wie folgt angepasst - \fullref{Anpassung der sshd_config}.

\begin{code}[H]
    \begin{minted}{bash}
        sudo vi /etc/ssh/sshd_config

        PasswordAuthentication no
        PermitRootLogin no
    \end{minted}
    \caption[Anpassung der sshd\_config]{Anpassung der sshd\_config}
    \label{Anpassung der sshd_config}
\end{code}

Um die neu gesetzten SSH-Zugriffsrechte zu aktivieren, ist ein Neustart des sshd-Services nötig.
Der sshd-Service wird mit dem folgenden Kommandozeilenbefehle neu gestartet:

\mintinline{bash}{sudo systemctl restart sshd}

Die SSH-Zugriffsrechte sind nun gültig.
Gültige Benutzer können sich jetzt nur noch über die SSH-Schlüssel mit dem Server verbinden.
Dem Root-Benutzer wird ein Remote-Login untersagt.\vgonlinezitat{sshaccess}

\textbf{Deaktivierung SELinux}

\ac{SELinux} ist eine Sicherheitsarchitektur für Linux-Systeme.
Mithilfe dieser Architektur können diverse sicherheitstechnische Prozesse und Zugriffe gezielt gesteuert und kontrolliert werden.
Entwickelt wurde das Konzept von der US-amerikanischen \ac{NSA}.
Im Jahre 2000 wurde SELinux schließlich als Open-Source freigegeben und 2003 in den Linux-Upstream-Kernel integriert.\vgonlinezitat{selinux}

SELinux bietet diverse Vorteile bei der Überwachung und Kontrolle von Prozesszugriffen.
Durch die enorme Komplexität können jedoch auch Zugriffsfehler entstehen, welche nur schwer zu lösen sind.
Eine manuelle Veränderung der SELinux-Konfiguration ist nur unter äußerster Vorsicht durchzuführen.
Flasche bzw. undurchdachte Einstellungen können zu Fehlen führen, welche nur sehr schwer wieder zu entfernen sind.
Von der manuellen Konfiguration SELinuxs wird daher abgeraten.

Für das folgende Projekt wird SELinux deaktiviert.
Der Status von SELinux kann mit dem folgenden Kommandozeilenbefehle abgefragt werden:\vgonlinezitat{selinuxno}

\mintinline{bash}{sestatus}

Um SELinux zu deaktiveren wird die SELinux-Konfigurationdatei wie folgt verändert - \fullref{Deaktivierung SELinux}

\begin{code}
    \begin{minted}{bash}
# This file controls the state of SELinux on the system.
# SELINUX= can take one of these three values:
#     enforcing - SELinux security policy is enforced.
#     permissive - SELinux prints warnings instead of enforcing.
#     disabled - No SELinux policy is loaded.
SELINUX=disabled
# SELINUXTYPE= can take one of these three values:
#     targeted - Targeted processes are protected,
#     minimum - Modification of targeted policy. Only selected processes are protected. 
#     mls - Multi Level Security protection.
SELINUXTYPE=targeted
    \end{minted}
    \caption[Deaktivierung SELinux]{Deaktivierung SELinux}
    \label{Deaktivierung SELinux}
\end{code}



\section{MariaDB-Datenbank}
Für die Sicherung der Anmeldedaten der einzelnen Studenten wird eine Datenbank benötigt.
Gleichzeitig werden hier auch alle weiteren Stunden spezifischen Daten hinterlegt.
Diese Datenbank wird direkt auf dem Server installiert und eingerichtet.
Die Software, welche zum späteren Zeitpunkt zur Ermittlung einer neuen Campus-Dual Note eingesetzt wird, erhält alle benötigten Informationen des Studenten aus dieser Datenbank.
Für das folgende Projekt wird eine Maria-Datenbank eingesetzt.

\textbf{MariaDB-Service installieren und aktivieren}

Für die Einrichtung einer MariaDB werden die folgenden Pakete heruntergeladen und anschließend installiert.

\mintinline{bash}{dnf install mariadb-server mariadb}

Im Anschluss wird der MariaDB-Service mit dem Linux-Tool \textit{systemctl} eingeschalten und dauerhaft aktiviert.
Hierfür wird der folgende Kommandozeilenbefehle verwendet.

\mintinline{bash}{systemctl enable --now mariadb}

Die Option \textit{enable} sorgt für die dauerhafte Aktivierung des Services.
Das bedeutet, dass der Service bei jedem Neustart des Servers automatisch und ohne weiteres Eingreifen hochgefahren wird.
Somit ist die Datenbank auch erreichbar, wenn der Server einmal neu hochfährt.
Jedoch wird der Service durch diese Option erst nach erstmaligen Neustart des Servers aktiv.
Die Option \textit{--now} sorgt für die Aktivierung des Services direkt nach der Eingabe des Kommandozeilenbefehls.
Der MariaDB-Service ist somit direkt aktiv und startet sich bei jedem Neustart des Servers automatisch.
\newpage

\textbf{mysql\_secure\_installation}

Die Installation der Maria-Datenbank wird mit dem folgenden Kommandozeilenbefehl eingeleitet.

\mintinline{bash}{mysql_secure_installation}

Der Kommandozeilenbefehl startet eine interaktive Installationsanleitung.
Die MariaDB wird entsprechend der ausgewählten Optionen installiert und eingerichtet.
Der folgende \literef{mysql_secure_installation} zeigt, die verwendetet MariaDB Installationskonfiguration.\vgonlinezitat{mariadb}

\begin{code}
    \begin{minted}{bash}
NOTE: RUNNING ALL PARTS OF THIS SCRIPT IS RECOMMENDED FOR ALL MariaDB
SERVERS IN PRODUCTION USE!  PLEASE READ EACH STEP CAREFULLY!

In order to log into MariaDB to secure it, we'll need the current
password for the root user.  If you've just installed MariaDB, and
you haven't set the root password yet, the password will be blank,
so you should just press enter here.
  
Enter current password for root (enter for none): 
OK, successfully used password, moving on...

Change the root password? [Y/n] Y
... Success!

Remove anonymous users? [Y/n] Y
... Success!
  
Disallow root login remotely? [Y/n] Y
... Success!
  
Remove test database and access to it? [Y/n] Y
... Success!
  
Reload privilege tables now? [Y/n] Y
... Success!
  
Cleaning up...
All done!  If you've completed all of the above steps, your MariaDB
installation should now be secure.
  
Thanks for using MariaDB!
    \end{minted}
    \caption[mysql\_secure\_installation]{mysql\_secure\_installation}
    \label{mysql_secure_installation}
\end{code}

\textbf{Anlegen der Datenbank}
Nach der erfolgreichen Installation MariaDBs wird eine Datenbank angelegt.
Mit dem folgenden Kommandozeilenbefehl verbindet sich der Root-Benutzer mit der MariaDB.

\mintinline{bash}{mysql -u root -p}

Unter der Option \textit{-u root} wird der MariaDB-Nutzer angegeben, welcher sich mit der Datenbank verbindet.

Mit der Option \textit{-p} wird eine interaktive Passwortabfrage erzeugt, welche die Verbindung autorisiert.

Ist die Verbindung aufgebaut, kann mit dem folgenden Befehle die benötigte Datenbank erstellt werden - \fullref{Anlegen der Datenbank}.\vgonlinezitat{datenbank}

\begin{code}
    \begin{minted}{bash}
[root@debian:~]# mysql -u root -p
Enter password:
Welcome to the MariaDB monitor.  Commands end with ; or \g.
Your MariaDB connection id is 48677
Server version: 10.3.31-MariaDB-0 Rocky8

Copyright (c) 2000, 2018, Oracle, MariaDB Corporation Ab and others.

Type 'help;' or '\h' for help. Type '\c' to clear the current input statement.

MariaDB [(none)]> CREATE DATABASE campusdual;
Query OK, 1 row affected (0.00 sec)

mysql> exit;
    \end{minted}
    \caption[Anlegen der Datenbank]{Anlegen der Datenbank}
    \label{Anlegen der Datenbank}
\end{code}


\textbf{Anlegen der Tabelle}

Nach dem erfolgreichen Anlegen der Datenbank \textit{campusdual} wird die benötigte Tabelle mit entsprechenden Attributen angelegt.
Hierfür werden die folgenden Befehle ausgeführt - \fullref{Anlegen der Tabelle}.\vgonlinezitat{tabelle}

\begin{code}[H]
    \begin{minted}{yaml}

mysql> CREATE TABLE `campusdual`.`User` ( `id` INT NOT NULL AUTO_INCREMENT ,  `name` CHAR(255) NOT NULL ,  `email` CHAR(255) NOT NULL ,  `password` TEXT NOT NULL ,  `hash` TEXT NULL ,  `mnr` INT(7) NULL ,  `notification` BOOLEAN NULL DEFAULT NULL ,    PRIMARY KEY  (`id`)) ENGINE = InnoDB;
Query OK, 1 row affected (0.00 sec)

mysql> exit;

    \end{minted}
    \caption[Anlegen der Tabelle]{Anlegen der Tabelle}
    \label{Anlegen der Tabelle}
\end{code}

\textbf{Anlegen der Datenbanknutzer}
\label{dbuser}

Das Arbeiten mit dem Root-Benutzer ist aus Sicherheitsgründen nicht zu empfehlen.
Daher wird ein zusätzlicher Benutzer angelegt, welcher im späteren Verlauf mit der MariaDB interagiert.
Ein MariaDB-Benutzer kann mit dem folgenden Befehlen erstellt werden - \fullref{Anlegen eines Nutzers}.\vgonlinezitat{userdb}

\begin{code}
    \begin{minted}{yaml}

mysql> CREATE USER 'campusdualuser'@'%' IDENTIFIED BY '**********';
Query OK, 1 row affected (0.00 sec)
GRANT ALL PRIVILEGES ON campusdual . * TO 'campusdualuser'@'%';
Query OK, 1 row affected (0.00 sec)
FLUSH PRIVILEGES;
Query OK, 1 row affected (0.00 sec)

mysql> exit;

    \end{minted}
    \caption[Anlegen eines Nutzers]{Anlegen eines Nutzers}
    \label{Anlegen eines Nutzers}
\end{code}

\textbf{Portfreigabe für Fernzugriff}

Die standardmäßige Konfiguration der MariaDB verhindert ein Fernzugriff auf die Datenbank.
Der Fernzugriff wird jedoch benötigt, damit die Stunden sich von jedem beliebigen Gerät mit beliebiger IP-Adresse auf der Webseite für den Service registrieren können.
Um die entsprechenden Ports innerhalb der Firewall freizugeben, werden die folgenden Kommandozeilenbefehle verwendet.

\mintinline{bash}{sudo firewall-cmd --zone=public --add-service=mysql --permanent}

\mintinline{bash}{sudo firewall-cmd --reload}

\newpage

\section{Apache-Server}
Für die Webseite der Registrierung, Anmeldung und der Service-Bedienung wird ein Apache-Server verwendet.
Dieser wird direkt auf dem Server installiert.

\textbf{Installation und Konfiguration}

Zunächst wird das benötigte Paket heruntergeladen und installiert.
Dazu wird der folgende Kommandozeilenbefehl verwendet.

\mintinline{bash}{sudo dnf install httpd}

Um den httpd-Service zu nutzen, muss dieser zunächst gestartet und aktiviert werden.
Hierfür wird der folgende Kommandozeilenbefehl verwendet.

\mintinline{bash}{sudo systemctl enable --now httpd}

Für die Überprüfung des Service-Zustandes steht der folgende Kommandozeilenbefehl zur Verfügung.

\mintinline{bash}{sudo systemctl status httpd}

Anschließend wird der Service für die Firewall freigegeben.
Der Service kann mithilfe des folgenden Kommandozeilenbefehls freigeschaltet werden.

\mintinline{bash}{sudo firewall-cmd --add-service=http --permanent}

Erst nun ist es möglich, auch von nicht lokalen IP-Adressen auf den Apache-Server zuzugreifen.
Um die Firewall-Einstellungen zu übernehmen, wird der Firewall-Service neu gestartet.
Der Firewall-Service wird mit dem folgenden Kommandozeilenbefehl neu gestartet.

\mintinline{bash}{sudo firewall-cmd --reload}

\textbf{Aufbau der Webseiten}

\textit{Registrierung:}\\
Auf der Register-Webseite werden die Daten des Studenten in die Datenbank übertragen, um sich für den Service der Benachrichtigung zu registrieren.
Der Student muss für die Registrierung die folgenden Daten angeben: Name, Passwort, E-Mail, Hashwert, Matrikelnummer.
Zunächst werden alle Daten validiert.
Sind die Daten korrekt, verbindet sich der in \literef{dbuser} erstellte MariaDB-Benutzers mit der Datenbank und trägt die entsprechenden Daten in einen neuen Datensatz ein.
Der Student ist nun registriert und kann den Benachrichtigungsservice aktivieren.

\newpage

\textit{Anmeldung:}\\
Auf der Anmelde-Webseite kann sich der Student mit seinen vorher registrierten Daten anmelden und wird damit auf die Webseite zur Bedienung des Services weitergeleitet.
Dafür muss der Student die hinterlegte E-Mail und das entsprechende Passwort eingeben.
Die Daten werden zunächst validiert.
Handelt es sich um valide Daten, werden diese mit den Daten aus der Datenbank abgeglichen.
Bei erfolgreicher Anmeldung wird ein Session-Cookie erzeugt und der Student auf die Webseite zur Bedienung des Services zur Benachrichtigung weitergeleitet.

\textit{Bedienung:}\\
Auf der Webseite zur Bedienung des Services kann der Student die Option zur Benachrichtigung via E-Mail aktivieren.
Dabei wird im Hintergrund der Wert des Attributes \textit{notification} des Datensatzes auf true gesetzt.
Standardmäßig wird dieser Wert innerhalb der Registrierung auf false gesetzt.
Falls gewünscht, kann der Student seinen gesamten Datensatz aus der Datenbank löschen.
Damit ist die Registrierung widerrufen und der Benachrichtigungsservice automatisch deaktiviert.

\section{phpmyadmin}
\textit{Phpmyadmin} ist eine quelloffene Software zur Verwaltung von SQL-Datenbanken im Webbrowser.
Diese wird verwendet, um die Datenbanken zu pflegen und zu warten.
Der größte Vorteil besteht dabei aus der benutzerfreundlichen Bedienoberfläche sowie der Möglichkeit auch remote auf die Datenbank zugreifen zu können.
Für die Installation von \textit{phpmyadmin} wird \textit{php} vorausgesetzt.
Daher wird auf dem Server zunächst \textit{php} heruntergeladen und installiert.

\mintinline{bash}{sudo dnf module list php}

Im folgenden Projekt wird die PHP-Version 7.4 verwendet.

\mintinline{bash}{sudo dnf module install php:7.4}

Anschließend kann mit der Installation \textit{phpmyadmins} vorgefahren werden.
Dafür wird in das folgende Verzeichnis gewechselt und mithilfe des Linux-Tools \textit{wget} die benötigte Zip-Datei heruntergeladen.

\mintinline{bash}{sudo cd /var/www}

\mintinline{bash}{wget https://files.phpmyadmin.net/phpMyAdmin/5.1.1/phpMyAdmin-5.1.1-all-languages.zip}

Die Zip-Datei wird entpackt und umbenannt.

\mintinline{bash}{unzip phpMyAdmin*}

\mintinline{bash}{mv phpMyAdmin* phpmyadmin}

Nun werden der Datei die entsprechenden Rechte vergeben.

\mintinline{bash}{sudo chown -R apache:apache phpmyadmin}

Die Datenbank ist nun über die Domain \textit{/phpmyadmin} erreichbar.
Der MariaDB-Nutzer kann sich nun mit den entsprechenden Anmeldedaten mit der Datenbank verbinden - siehe \literef{Anmeldung für phpmyadmin}.

\bild[1]{phpmyadmin.png}{Anmeldung für phpmyadmin}{Anmeldung für phpmyadmin}