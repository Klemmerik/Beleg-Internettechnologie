\chapter{Benachrichtigungs Software für neue Noten}
Im folgenden Kapitel wird erläutert, wie die Software zur automatischen Benachrichtigung konzipiert ist und wie diese auf den Server implementiert wird.
Die BA-Glauchau Vorschriften verbieten zum jetzigen Zeitpunkt die Verwendung realer Daten des Portals \textit{Campus Dual}.
Für das folgende Kapitel wird daher auf Demodaten zurückgegriffen.

\section{Konzeptionierung}
Das Grundkonzept der Software ist wie folgt aufgebaut.
Ein Skript greift auf die Datenbank zu und entnimmt sich die benötigten Daten des Studenten, um an die von der BA-Glauchau bereitgestellte JSON-Datei zu gelangen, worin sich die Informationen des aktuellen Notenstandes befindet.
Die dafür benötigten Daten bestehen aus dem hinterlegten Hashwert und der Matrikelnummer des Studenten.
Mit diesen Informationen wird mit dem Linux-Tool \textit{wget} eine Abfrage der aktuellen JSON-Datei durchgeführt.
Die Abfrage wird wie folgt ausgeführt:

\mintinline{bash}{wget https://selfservice.campus-dual.de/dash/getexamstats?user=4004444&hash=9081f....}

Der Inhalt der Datei ist in \literef{Demo JSON-Datei} abgebildet.

\begin{code}
    \begin{minted}{bash}
        EXAMS	25
        SUCCESS	19
        FAILURE	2
        WPCOUNT	5
        MODULES	19
        BOOKED	0
        MBOOKED	19
    \end{minted}
    \caption[Demo JSON-Datei]{Demo JSON-Datei}
    \label{Demo JSON-Datei}
\end{code}

Der aktuelle Notenstand (SUCCESS) wird aus dieser Datei ausgelesen und mit dem Notenstand verglichen, welcher sich im MariaDB Datensatz des Studenten befindet.
Ist der Notenstand der JSON-Datei größer als der, in der Datenbank hinterlegten Notenstand, ist demnach eine neue Note in \textit{Campus Dual} verfügbar.
Trifft dieser Fall zu, wird eine E-Mail an den Studenten versendet.
Die E-Mail-Adresse ist dabei die E-Mail, welche der Student bei der Registrierung verwendet hat.

\newpage
\section{Umsetzung}
\textbf{exams.py}
\begin{code}[H]
    \begin{minted}{python}
        import json
        import requests as requests
        
        from email_notifier import generate_email
        
        
        def get_userdata():
            with open("config/campus_dual.json") as conf_file:
                user_data = json.load(conf_file)
            return user_data.get('username'), user_data.get('userhash')
        
        
        def get_diff(username: str, userhash: str):
            url = "https://selfservice.campus-dual.de/dash/getexamstats?user=" + username + "&hash=" + userhash
            if requests.get(url, verify=False).status_code != 200:
                print("Something went wrong")
                return None
        
            # Open old stats and save in exam_stats
            with open("config/examstats.json", mode='r') as file:
                exam_stats = json.load(file)
        
            # Get latest stats
            new_stats = requests.get(url, verify=False).json()
        
            # Look for change
            if exam_stats != new_stats:
                x = None
                if new_stats.get('EXAMS') > exam_stats.get('EXAMS'):
                    x = 'a'
                    if new_stats.get("SUCCESS") > exam_stats.get('SUCCESS'):
                        x = 'ab'
                    elif new_stats.get("FAILURE") > exam_stats.get('FAILURE'):
                        x = 'ac'
                elif new_stats.get('BOOKED') > exam_stats.get('BOOKED'):
                    x = 'd'
                if x is not None:
                    generate_email(x)
                with open("config/examstats.json", mode='w') as file:
                    json.dump(new_stats, file)
    \end{minted}
    \caption[exams.py]{exams.py}
    \label{exams.py}
\end{code}

\textbf{main.py}
\begin{code}[H]
    \begin{minted}{python}
        from exams import get_userdata, get_diff
        if __name__ == '__main__':
            username, userhash = get_userdata()
            get_diff(username, userhash)
    \end{minted}
    \caption[main.py]{main.py}
    \label{main.py}
\end{code}

\textbf{email\_notifier.py}
\begin{code}[H]
    \begin{minted}{python}
from fpdf import FPDF

def send_email():
    pass

def generate_email(variante: str):
    email_pdf = FPDF()
    email_pdf.add_page()
    email_pdf.set_font("Arial", size=15)
    if 'a' in variante:
        email_pdf.cell(200, 10, txt="Neue Prüfungsergebnisse!", ln=1, align='C')
        if 'b' in variante:
            email_pdf.cell(200, 10, txt="Prüfung bestanden!", ln=2, align='C')
        if 'c' in variante:
            email_pdf.cell(200, 10, txt="Durchgefallen!", ln=2, align='C')
        else:
            email_pdf.cell(200, 10, txt="Bitte öffne Campus Dual!", ln=2, align='C')
        email_pdf.output("Email.pdf")
    elif 'd' in variante:
        email_pdf.cell(200, 10, txt="Neue Anmeldungen verfügbar!", ln=1, align='C')
        email_pdf.output("Email.pdf")
    \end{minted}
    \caption[main.py]{main.py}
    \label{main.py}
\end{code}
