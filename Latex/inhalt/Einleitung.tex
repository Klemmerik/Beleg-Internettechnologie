\chapter{Einleitung}

\section{Problemstellung}
Die Berufsakademie Glauchau verwendet das Verwaltungstool \textit{Campus Dual}, um ihren Studenten diverse organisatorische Informationen zur Verfügung zu stellen.
Dazu gehört unter anderem die Bereitstellung der jeweiligen Unterrichtspläne, Informationen zu bevorstehenden Modulprüfungen, studienrelevante Dokumente sowie die Noteneinsicht abgeschlossener Module.
Zum jetzigen Zeitpunkt ist die Einsicht der Noten von abgeschlossenen Modulen, allein über das Verwaltungstool \textit{Campus Dual} möglich.
Nach jeder Prüfungsphase und der darauffolgenden Auswertung werden die Noten der jeweiligen Prüfungen den Studenten zur Einsicht auf \textit{Campus Dual} bereitgestellt.
Die Zeitspanne vom Prüfungstag bis hin zur letztendlichen Bekanntgabe der Noten variiert dabei stark.
Eine Benachrichtigung über die Bekanntgabe einer neuen Note in \textit{Campus Dual} existiert zum jetzigen Zeitpunkt nicht.
Die einzige Möglichkeit besteht darin, das Verwaltungstool \textit{Campus Dual} in regelmäßigen Zeitabständen zu besuchen und zu Überprüfung, ob eine neue Note bekannt gegeben wurde.
Da unter anderem bis zur Bekanntgabe einer neuen Note mehrere Wochen oder auch Monate verstreichen können, kann sich dies als sehr zeitaufwändig entpuppen.
Um dieses Problem entgegenzuwirken, wird eine Möglichkeit zur automatisierten Benachrichtigung neuer \textit{Campus Dual} Noten untersucht.


\section{Zielstellung}
Ziel der vorliegenden Arbeit ist die Untersuchung einer Möglichkeit zur automatisierten Benachrichtigung neuer \textit{Campus Dual} Noten.
Studierende sollen individuell über ihre neuen Noten benachrichtigt werden und somit von der zwangsläufigen Einsicht \textit{Campus Duals} entbunden werden.
Den Studierenden wird die Möglichkeit geboten, den Service zur Benachrichtigung neuer Noten selber aktivieren oder auch deaktivieren zu können.
Dazu wird eine Webseite zur Verfügung gestellt, worauf sich die Studierenden der \textit{BA-Glauchau} für den Service anmelden können.
Die Benachrichtigung über eine neue Note wird in Form einer versendeten E-Mail bereitgestellt.
Die E-Mail ist dabei frei wählbar.
Um die Informationssicherheit zu gewährleisten, kommt der gesamte Benachrichtigungsprozess ohne das individuelle Passwort des \textit{Campus Dual} Nutzers aus.






 
